%%%%%%%%%%%%%%%%%%%%%%%%%%%%%%%%%%%%%%%%%%%%%%%%%%%%%%%%%%%%%%%%%%%%%%%%%%%%%%%%
% Author : [Name] [Surname], Tomas Polasek (template)
% Description : Seventh exercise in the Introduction to Game Development course.
%   It deals with the creation of a Game Design Document, presenting a short 
%   pitch for a potential game project.
%%%%%%%%%%%%%%%%%%%%%%%%%%%%%%%%%%%%%%%%%%%%%%%%%%%%%%%%%%%%%%%%%%%%%%%%%%%%%%%%

\documentclass[a4paper,10pt,english]{article}

\usepackage[left=2.50cm,right=2.50cm,top=1.50cm,bottom=2.50cm]{geometry}
\usepackage[utf8]{inputenc}

% Hyper-Text References
\usepackage{hyperref}
\hypersetup{colorlinks=true, urlcolor=blue}

% Drawing Images and Graphs
\usepackage{tikz}
\usepackage{pgfplots}

% Page Utilities
\usepackage{graphicx}

% Image Sub-Captions
\usepackage{subcaption}

\newcommand{\ph}[1]{\textit{[#1]}}

\title{%
Game Pitch Document%
}
\author{%
Petr Šebela (xsebelp00)%
}
\date{}

\begin{document}

\maketitle
\thispagestyle{empty}

{%
\large

\begin{itemize}

\item[] \textbf{Title:} Primal Equation

\item[] \textbf{Genre:} open-world sandbox

\item[] \textbf{Style:} 3D cartoon

\item[] \textbf{Platform:} PC [Windows, Linux]

\item[] \textbf{Market:} Anyone interested in science and process of using it

\item[] \textbf{Elevator Pitch:} Use science and determination to revive what was once lost to time

\end{itemize}

}

\section*{\centering The Pitch}

\subsection*{Introduction}
Primal Equation is a game about using science to achieve goals. The game world closely resembles the real one, at least the physics part of it. The player will be able to create almost anything, supposing they have a basic understanding of how it works.

\subsection*{Background}
I love the idea of using science to propel civilization forward. This journey led me to manga called \textit{Dr. Stone}, which captures the journey of a young scientist who reintroduces modern science to a world that all but forgot about it. This inspired me to create game just about that. The process of reinventing science. Another part that inspired me, is Minecraft's Create mod. It is a mod about building mechanical machines and automation. It manages to explain complex mechanism really well and enabling the player to do anything they want.

\subsection*{Setting}
The player wakes up to a strange world, it is nearly without humans and with primitive technology. They (the player) have no recollection of who they are, all they remember is their scientific knowledge. Their goal is to explore this ancient world, learn who they are and what happened to the world, and reintroduce modern science to it.

\subsection*{Features}
Main focus point of Primal Equation is its crafting system. The premise is to shift the focus from the result to the actual process. Crafting process will be intricate, requiring the player to take non linear approach and create other items in support of their initial goal. Using diagetic UI as much possible to keep the player immersed in the game. Creating will require gathering resources. This process will be initially done by hand, automating it, as the player progresses throughout the game. Everything will be accessible to the player from the beginning, if they have the knowledge to do so. There will be crafting roadmaps for individual items to help guide new players in this complex crafting system.

\subsection*{Genre}
The game is open-world sandbox. This puts the crafting part of the game to the main stage. The player is free to accomplish their goal in any way that they see fit. Open-world will be driven by procedural generation with parts of it being made by hand. This will enables us to script a simple story that will motivate the player to continue on their endeavors of reintroducing modern science to this primitive world.

\subsection*{Platform}
The game needs to strike balance between being too complex and too simple. Using anything other than keyboard and mouse to control it is probably out of the question. Using gamepad to control it would require significant simplification of certain systems. While it could be possible to use the gamepad, it certainly would be difficult to implement this without sacrificing some of the initial ideas. Because of this, the Primal Equation will be limited to PC.

\subsection*{Style}
The Primal Equation will feature cartoonish graphics with large amount of detail. This will hopefully invoke creativity and playfulness in the player. Everything that the player makes will be distinct from structures made by people already in the world. The player has only basic tools at hand, so quality and design of said items should correspond to this fact.

\begin{figure}
    \centering
    \includegraphics[width=0.75\linewidth]{IMG_0175.jpg}
\end{figure}

% \ph{Replace all text in this section with the game design pitch...}

% \subsection*{Instructions}

% In this exercise, you assume the role of an enterprising game developer who has a great idea for a new game -- \emph{The Game}. You are tasked with the creation of a short \emph{Game Design Document}. That is, a \emph{pitch} of The Game's main idea to potential \emph{investors} or \emph{leadership} of a game studio. If you have your own ideas for a game, this is a great opportunity to develop them further. Alternatively, you can also choose an already existing game, but I recommend trying to come up with your own idea first. 

% Within this template, you will find placeholders and hints \ph{like this one}, which you should read and replace with your own text. You can use any means of expression you deem appropriate -- text, reference images from other games, sketches, diagrams, tables, graphs, etc. Remember the goal: ``selling'' your idea so that you get the opportunity to actually \emph{make} The Game. Keep it brief and to the point. The length of your final document \textbf{shouldn't exceed 2-3 pages}. Following are example sections and pointers as to what they could contain. However, the document structure is certainly not set in stone. Feel free to modify it as necessary.

% \subsection*{Introduction}
% This should be the core of your pitch. Describe what \emph{exactly} it is you want to make. What is \emph{important}, what makes your game \emph{special}. All in one paragraph (50 words max).

% \subsection*{Background}
% What lead you to The Game's basic idea? What are the inspirations -- other games (even physical), sports, events, etc. Is it a continuation of some long-going traditional genre? Are you trying to bring back something that worked in the past?

% \subsection*{Setting}
% Describe the setting of your game. Is your game \emph{narrative-based}? You should detail the basic plot here. Cover the character of your \emph{protagonist} and their interaction with the environment. Will your story be \emph{interactive}? You can put some example dialogues and possible choices here as well. Even games with \emph{light} or \emph{no narrative} take place in some kind of universe.

% \subsection*{Features}
% What are the main \emph{selling points} of your game? Think about the \emph{target market} and \emph{market values} of your game. What makes it unique among other, already existing games? Why would players want to play \emph{your} game instead of some other? You can use a bullet point list or combine it with a \emph{value graph}.

% \begin{figure}[h]

% \centering
    
% \begin{tikzpicture}[remember picture]%
% \begin{axis}[
%     domain=0:1, 
%     clip=false, 
%     ymin=0, xmin=0, ymax=4.1, xmax=12, 
%     xtick={1,2,...,11}, 
%     yticklabels={}, 
%     xticklabels={Cyberpunk Plot, Stealth Gameplay, Reactive Story, Atmospheric World, 1st Person Combat, RPG Progression, Philosphical Themes, Advanced Graphics, Supernatural Powers, Art Deco Style, Sandbox Creativity}, 
%     xticklabel style={yshift={-mod(\ticknum, 2) * 2em}, text width=1.5cm, font=\small}, 
%     y tick style={draw=none}, 
%     x axis line style={|-|}, 
%     y axis line style={draw=none}, 
%     axis lines=middle, 
%     width=\linewidth, 
%     height=0.3\paperheight
% ]
%     \addplot [ultra thick, mark=square*, mark options={scale=2,solid}] coordinates {
%         (1, 2.7)
%         (2, 1.3)
%         (3, 3.1)
%         (4, 3.0)
%         (5, 1.3)
%         (6, 1.8)
%         (7, 2.9)
%         (8, 0.8)
%         (9, 0.1)
%         (10, 0.1)
%         (11, 0.1)
%     } node[above, yshift=15pt, pos=0] {\textbf{Deus Ex}};
    
%     \addplot [ultra thick, dashed, mark=triangle*, mark options={scale=2,solid}] coordinates {
%         (1, 0.1)
%         (2, 0.5)
%         (3, 0.7)
%         (4, 2.8)
%         (5, 2.9)
%         (6, 2.0)
%         (7, 3.0)
%         (8, 3.2)
%         (9, 3.1)
%         (10, 3.1)
%         (11, 0.1)
%     } node[above, yshift=5pt, xshift=-10pt, pos=0] {\textbf{BioShock}};
    
%     \addplot [ultra thick, dotted, mark=diamond*, mark options={scale=2,solid}] coordinates {
%         (1, 0.1)
%         (2, 0.1)
%         (3, 0.1)
%         (4, 0.1)
%         (5, 0.3)
%         (6, 0.1)
%         (7, 0.1)
%         (8, 1.2)
%         (9, 0.1)
%         (10, 0.1)
%         (11, 3.1)
%     } node[above, yshift=5pt, xshift=-5pt, pos=1, text width=1cm, align=center] {\textbf{Garry's Mod}};
    
% 	\draw [] (axis cs:{0,1}) -- (axis cs:{12,1});
% 	\draw [] (axis cs:{0,2}) -- (axis cs:{12,2});
% 	\draw [] (axis cs:{0,3}) -- (axis cs:{12,3});
% 	\draw [] (axis cs:{0,4}) -- (axis cs:{12,4});
% \end{axis}
% \end{tikzpicture}

% \caption{Example value graph for \emph{Deus Ex}, \emph{BioShock}, and \emph{Garry's Mod}.}
% \label{Fig:ValueGraph}

% \end{figure}

% \subsection*{Genre}
% Specify the genre of The Game. Be \emph{clear}, but be sure to note on the \emph{nuances} which set your game apart from others within the same genre. 

% \subsection*{Platform}
% What are the platforms you plan to release The Game on? Do you have a core set in mind? Are you going to release versions for other platforms later?

% \subsection*{Style}
% Here, you can provide a visualization of what The Game would look like. Don't have concept artist at hand? Use diagrams, schemes, or illustrate on images from already existing games. It is time to dust off your \emph{Microsoft Paint} skills!

% \begin{figure}[h]

% \centering

% \begin{subfigure}{0.29\linewidth}
% \includegraphics[width=\linewidth]{example-image-a}
% \captionof{figure}{Style Exhibit 1a.}
% \label{Fig:Style1A}
% \end{subfigure}\hfill
% %
% \begin{subfigure}{0.29\linewidth}
% \includegraphics[width=\linewidth]{example-image-b}
% \captionof{figure}{Style Exhibit 1b.}
% \label{Fig:Style1B}
% \end{subfigure}\hfill
% %
% \begin{subfigure}{0.29\linewidth}
% \includegraphics[width=\linewidth]{example-image-c}
% \captionof{figure}{Style Exhibit 1c.}
% \label{Fig:Style1C}
% \end{subfigure}

% \end{figure}

% \subsection*{Formatting \& Submission}

% Your submission should follow a similar \textbf{structure} to this template. You can either use the provided \LaTeX\ template or roughly replicate it in some other text processing software. The format of the section ``\emph{The Pitch}'' is left up to you. For inspiration, see the game design documents included on the \href{http://cphoto.fit.vutbr.cz/ludo/courses/izhv/exercises/e7/}{assignment page}, or the lecture dedicated to \href{http://cphoto.fit.vutbr.cz/ludo/courses/izhv/lectures/l12/}{game development}. The only accepted document format is \textbf{pdf}. You can submit the pdf by following the submission guidelines detailed on the \href{http://cphoto.fit.vutbr.cz/ludo/courses/izhv/exercises/sub/}{course's website}. 

\end{document}
